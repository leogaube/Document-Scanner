\documentclass{scrartcl}
\usepackage[ngerman, english]{babel}
\usepackage{rwukoma}
\usepackage[pdfusetitle]{hyperref}
\usepackage{lipsum,caption}

\title{report: document scanner}
\author{Leopold Gaube and Florian Betz}
\date{\today}

\begin{document}
	\maketitle
	\tableofcontents

	\clearpage

	\section{Erster Abschnitt}

		Das ist ein Test der verschiedenen Textstile:
		\emph{hervorgehoben}, \textit{kursiv}, \textbf{fett} sowie
		\textit{\textbf{fett und kursiv}}.
		Bei \texttt{nichtproportionalem} Text muss man ein bisschen
		auf die Zeilenenden achten, da \LaTeX{} in diesem keine
		Silbentrennung durchführt.

		Auch dafür werden aber die üblichen Textstile unterstützt:
		\begin{itemize}
			\item \texttt{normal}
			\item \texttt{\textbf{fett}}
			\item \texttt{\textit{kursiv}}
			\item \texttt{\textbf{\textit{beides}}}
		\end{itemize}

		\subsection{Erster Unterabschnitt}

			Das RWU-Logo kann einfach über das Kommando
			\verb?\rwulogo? eingebunden werden und ist in
			Abbildung~\ref{fig:logo} zu sehen.

			\begin{figure}[hb]
				\centering
				\rwulogo[width=0.5\columnwidth]
				\caption{Das RWU-Logo}
				\label{fig:logo}
			\end{figure}

		\subsection{Zweiter Unterabschnitt}

			\lipsum[1]

			\subsubsection{Unterunterabschnitt}

			\lipsum[2-5]
\end{document}
