\documentclass[bibliography=totoc]{scrartcl}
\usepackage[ngerman, english]{babel}
\usepackage{rwukoma}
\usepackage[pdfusetitle]{hyperref}
\usepackage{lipsum,caption}
\usepackage{acronym} 

\title{Document Scanner}
\author{Leopold Gaube and Florian Betz}
\date{\today}

\begin{document}
	\maketitle
	\tableofcontents

	\clearpage

	\section{Requirements}

		There are already many document scanners on the market that can digitalize a paper or similar. 
		The quality of the document scanning depends on the used algorithms.
		To garantues a good scan, a perspective transformation should be used.
		
	    A comfortable library to use computer vision functions is \ac{OpenCV}.
	    \ac{OpenCV} is an open source computer vision and machine learning software library. \cite{OpenCV}
	    You can use OpenCV with the programming languages C, C++, Python or Java.
	    
	    

    \section{Algorithm Overview}
	Our algorithm for scanning a document in an image consists of three main parts, each with its own challenges.
	
	First, we have to localize the document in the image. 
	Then we use a perspective transform from the corner points of the document to the entire span of the image in order to obtain a top-down view without perspective distortions.
	Finally, we use a binary filter to distinguish text from background and save the resulting image as our scanned document.

	\section{Challenges}
	It is fairly straight forward if we work with an image like in Figure 1a.
	However, we also want our algorithm to work under bad lighting conditions [Fig. 1b], taken from an extreme angle [Fig. 1c], with a document on a low contrast background [Fig. 1d].
	Of course the final image will suffer in quality compared to a document taken under near-optimal conditions, but our goal is to still obtain a readable document.
	
	\section{Document Localization}
	Some smartphone scanner apps require the user to mark the four corner points of the document manually. 
	However, we want to automate this process by detecting the corner points using Computer Vision.
	Detecting the corner points accurately and consistantly is probably hardest and most crutial step of the entire algorithm. 
	
		\subsection{Preprocessing}
		Before we do Edge Detection, it is advisable to smooth the image in order to get rid of noise. Otherwise this noise may result in detection of edges which may negatively influence the algorithms performance.
		--> Gaussian Blur
		\subsection{Edge Detection}
		--> Canny Edge Detection (Sobel instead?)
	
		\subsection{Contours}
		ToDo: Explain contours

		We can assume that in our edge image there is a contour that matches the outline of our document. 
		It is ok, if we miss the exact location of the document corners by a couple of pixels, but detecting the wrong contour will make the resulting document unreadable.

		Detecting contours is easy with OpenCV using the cv2.findContours(...) method.
		However, we have to distinguish the one corresponding to our document from all the other contours in the image.
		Other contours may include text, graphs and pictures on the document itself or even objects e.g. a stapler on the desk. 
		The contour we are looking for is one of the largest by area and we should be able to approximate it with only four points.
		Both of these assumptions will help us to find the right one, as it is highly unprobable another contour meets both criteria. 
		We will just pick the largest contour that can be approximated using four points as show by the following pseudo code:
		
	\section{Perspective Transformation}
	sort points --> pseudo code
	top-down view
	
	\section{Thresholding}
		\subsection{Simple Thresholding}
		For the last step of our project we convert the scanned document into a binary image, thus each pixel is supposed to be either black for text and graphs or white for the background.
		The easiest way to achieve this is by converting the color image into grayscale and setting a threshold value. 
		Any pixel value below this threshold would become black, pixel values above would be set to white.

		\subsection{Adaptive Thresholding}
		However, there is a problem with this simple thresholding approach, as it performs poorly under bad lighting conditions as can be seen in [Fig. 3a].
		When taking pictures of documents, a user may cast a shadow with their camera or mobile phone onto the document. 
		This can be problematic if a region in the shadows has lower pixel values for the background than the text in a well-lit region.
		In such a scenario, it would be impossible to find a static threshold that works well for the entire document.
		So instead we need an adaptive approach that looks at a set of neighboring pixels and chooses a threshold for this local region dynamically.

	\section{known problems and possible solutions}
	Our algorithm works quite well and consistently even under difficult lighting conditions or when the picture of the document was taken from an unusual angle.
	However, we also observed some use cases where it fails:
	
	If the document is placed on a low-contrast background (e.g. a white desk), our algorithm has problems detecting an edge where the document outline should be.
	In some cases it may already help to do more preprocessing e.g. contrast enhancement.
	Currently, our program uses the Canny Edge Detector only once with a fixed lower and upper threshold as this works best for the majority of images.
	Whenever the document localization fails, we could try to repeat the process with lower threshold. 
	This will introduce more noise in the edge image, but it might also be enough to detect the document outline.

	Another situation in which our program works poorly is with bent document corner. 
	This may break the localization algorithm, because the detected contour would - theoretically - have to be approximated using five instead of four corner points.
	In some cases with minor bent corners this still works ok, as we allow for an error of 3\% when approximating the document contour(see Fig. 4a) which may still result in a four point contour approximation. 
	The resulting image will definitely have some small distortion due to the inaccurate localization of the bent corner, but at least the algorithm does not fail entirely.
	The same problem arises if the user is not careful while taking the image and accidentally cutting off a single corner from the picture composition.
	
	For the last two problematic use cases, we already have a possible solution in mind: 
	Whenever the document localization fails, we could try a completely different approach using a Hough Transformation. 
	Hough Transformations are very good at detecting straight lines in edge images. 
	Detecting the four most prevalent lines (with maximum-supression) in an edge image, corresponds to detecting the most likely document edges.
	The intersections of any two lines give us the location of each document corner. 
	A bent/hidden corner or even a corner outside of the image composition should be less of a problem when using the Hough Transformation approach.
	Nevertheless, it has its own downside, because the most prevalent lines have to be document edges and not for instance the edge of a desk.


\section*{List of Acronyms} 
\addcontentsline{toc}{section}{List of Acronyms}

\begin{acronym}[....]
    \acro{OpenCV}{Open Source Computer Vision Library}
\end{acronym}
\bibliographystyle{alpha}			
\bibliography{literature}
\end{document}
