\documentclass[bibliography=totoc]{scrartcl}
\usepackage[ngerman, english]{babel}
\usepackage{rwukoma}
\usepackage[pdfusetitle]{hyperref}
\usepackage{lipsum,caption}
\usepackage{acronym} 

\title{Document Scanner}
\author{Leopold Gaube and Florian Betz}
\date{\today}

\begin{document}
	\maketitle
	\tableofcontents

	\clearpage

	\section{Requirements}

		There are already many document scanners on the market that can digitalize a paper or similar. 
		The quality of the document scanning depends on the used algorithms.
		To garantues a good scan, a perspective transformation should be used.
		
	    A comfortable library to use computer vision functions is \ac{OpenCV}.
	    \ac{OpenCV} is an open source computer vision and machine learning software library. \cite{OpenCV}
	    You can use OpenCV with the programming languages C, C++, Python or Java.
	    
	    

    \section{}
	Our algorithm for scanning a document in an image consists of three main parts, each with its own challenges.
	
	First, we will have to localize the document in the image. 
	Then we will use a perspective transform from the corner points of the document to the entire span of the image in order to obtain a top-down view without any distortions.
	Finally, we will use a binary filter to distinguish text from background.
		\subsection{Document Localization}
		Some smartphone scanner apps require the user to mark the four corner points of the document manually. 
		However, we want to automate this process by detecting the corner points using Computer Vision.

		Detecting the corner points accurately and consistantly is probably hardest and most crutial step of the entire algorithm.
		
		\subsection{Perspective Transformation}
		top-down view
		\subsection{Post-Processing}
		binary image

		simple thresholding vs adaptive Thresholding
			
\section*{List of Acronyms} 
\addcontentsline{toc}{section}{List of Acronyms}

\begin{acronym}[....]
    \acro{OpenCV}{Open Source Computer Vision Library}
\end{acronym}
\bibliographystyle{alpha}			
\bibliography{literature}
\end{document}
